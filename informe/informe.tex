\documentclass[a4paper,spanish] {article} 

\usepackage{amsmath}
\usepackage{amsfonts}
\usepackage{listings}
\usepackage{amssymb}
\pagestyle{empty}

\newcommand{\real}{\ensuremath{\mathbb{R}}}

\usepackage [spanish] {babel} 
\usepackage [latin1]{inputenc}
\usepackage{graphicx}
\usepackage{caratula}
\usepackage{subfig}
\usepackage{dsfont}
\usepackage{algorithm}
\usepackage{amsmath}
\usepackage{algorithmic}
\usepackage{sidecap}
\usepackage{ulem} 

\parindent = 0 pt
\parskip = 11 pt

\usepackage{amsmath}
\usepackage{amsfonts}
\usepackage{amssymb}
\pagestyle{empty}



%links
\usepackage{hyperref}
\hypersetup{
    colorlinks,%
    citecolor=blue,%
    filecolor=blue,%
    linkcolor=blue,%
    urlcolor=blue
}

\def\dashuline{\bgroup 
\ifdim\ULdepth=\maxdimen  % Set depth based on font, if not set already
\settodepth\ULdepth{(j}\advance\ULdepth.4pt\fi
\markoverwith{\kern.15em
\vtop{\kern\ULdepth \hrule width .3em}%
\kern.15em}\ULon}


\addtolength{\oddsidemargin}{-1in}
\addtolength{\textwidth}{2in}


\begin{document}
\pagestyle{headings}

\newpage

\materia{Teor\'ia de Lenguajes}
\submateria{Segundo Cuatrimestre del 2011}
\titulo{Trabajo Pr\'actico}


\integrante{Ezequiel Castellano}{161/08}{ezequielcastellano@gmail.com}
\integrante{Mariano Semelman}{143/08}{ marianosemelman@gmail.com}

\maketitle

\newpage
\tableofcontents
\newpage
\newpage
\section{Gram\'atica}

Optamos por describir la gram\'atica de manera tal que no tenga conflictos de ning�n tipo (reduce/reduce o shift/reduce). A su vez no perdimos ni agregamos poder de expresividad, ya que de esta manera podemos representar las mismas cadenas que antes. 

La gram\'atica utilizada es la siguiente:

\lstinputlisting{./gramatica}
\bigskip

\section{Implementaci\'on}

La implementaci�n fue realizada en el lenguaje C++, utilizando la herramienta Bison. 

Para la implementaci�n utilizamos la estructura Aut�mata, que internamente es un grafo dirigido y etiquetado, por eso su nombre LDGraph (Labeled Derived Graph).

En la clase Aut�mata se implementaron todos los m�todos necesarios para poder concatenar aut�matas, demterminizarlos, mostrarlos, aplicar un operador o matchear un string. 

Por otra parte la clase LDGrafo consta de un estado de inicial, un conjunto de estados finales y un vector de transiciones como estructura. 

Donde las transiciones son un mapa donde la clave es la etiqueta de la transicion y el significado es el conjunto de estados que se pueden alcanzar por esa transci�n.

El grafo permite ver si se puede mover a un estado consumiendo un caracter, moverse por una transicion, determinizarse y ver si se encuentra en un estado de aceptaci�n, entre sus operaciones m�s destacables. 

Para ver m�s detalle sobre las estructuras en el ap�ndice se encuentra detallada la implementaci�n.

\section{Informaci\'on y requerimientos de software}

En esta secci\'on se indicar\'an versiones, herramientas, compiladores y todo lo necesario para la realizaci\'on del trabajo pr\'actico.

\subsection{Requerimientos}

El trabajo pr\'actico fue implementado sobre Ubuntu 11.05. Las siguientes instrucciones son v�lidas en cualquier sistema operativo Linux. De utilizar otro sistema operativo o realizar los pasos an�logos.

Es necesario tener instaladas las librerias para compilar en C++ y Bison.

Para instalar las librerias de Build Essencial (incluye librerias para compilar C++): sudo apt-get install build-essential manpages-dev.

Para instalar las librerias de Bison: sudo apt-get install bison bison-doc.

Opcional: Para poder renderizar los archivos .dot con el ejecutable dot. 

Para instalar las librerias para renderizar dot: sudo apt-get install graphviz.

\subsection{Compilar}

Utilizar el MakeFile provisto con el c\'odigo, el cual se encuentra detallado en el ap\'endice.

\begin{itemize}

 \item Para compilar solamente realizar: make.

 \item Para compilar los tests realizar: make build-test.

 \item Para compilar los tests y ejecutarlos: make test.

 \item Para remover los archivos compilados: make clean. 

 \item Para generar el gr�fico de la gram�tica: make bison (La imagen se encuentra en ./src/graph/grammar.png).

\end{itemize}

\subsection{Ejecutar}

Para ejecutar el programa se deber� estar posicionado en el directorio donde se encuentre el ejecutable, es decir el directorio donde se compil�.

El archivo ejecutable se llama grep-line y para utilizarlo deber� escribirse la siguiente sentencia: ./grep-line regexp [file] 

El primer par�metro es la expresion regular, mientras que el segundo par�metro opcional es un archivo con las cadenas a matchear, en caso de no utilizarse matchea por entrada estandard.

Para ver gr�ficos del aut�mata generado se deber� agregar el flag -i. 

//TODO: �Completar como armamos las im�genes?

\section{Casos de Prueba}

//TODO: Estos fueron los que usamos al comienzo para ver que la gram\'atica sea v\'alida, falta completar en que casos la deber\'ia aceptar y en cuales no. 

Las casos de prueba utilizados para validar la gram\'atica fueron los siguientes:

\lstinputlisting{../src/tests/test-regexp}
\bigskip

\section{Resultados}

//TODO: Completar cuales fueron los resultados obtenidos. 

\section{Conclusiones}

//TODO: Completar las conclusiones. 

Nos fue muy �til para chequear que la gram\'atica no tenga conflictos el tener una herramienta como Bison. 

\section{Ap\'endice}

\subsection{MakeFile}

\lstinputlisting[language=make]{../src/Makefile}
\bigskip

\subsection{Automata}

\subsubsection{Headers}
\lstinputlisting[language=C++]{../src/automata.hpp}
\bigskip

\subsubsection{Impementaci�n}
\lstinputlisting[language=C++]{../src/automata.cpp}
\bigskip


\subsection{Grafo}

\subsubsection{Headers}
\lstinputlisting[language=C++]{../src/grafo.hpp}
\bigskip

\subsubsection{Impementaci�n}
\lstinputlisting[language=C++]{../src/grafo.cpp}
\bigskip

\end{document}
