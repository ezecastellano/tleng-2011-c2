\documentclass[a4paper,spanish] {article} 

\usepackage{amsmath}
\usepackage{amsfonts}
\usepackage{listings}
\usepackage{amssymb}
\pagestyle{empty}

\newcommand{\real}{\ensuremath{\mathbb{R}}}

\usepackage [spanish] {babel} 
\usepackage [latin1]{inputenc}
\usepackage{graphicx}
\usepackage{caratula}
\usepackage{subfig}
\usepackage{dsfont}
\usepackage{algorithm}
\usepackage{amsmath}
\usepackage{algorithmic}
\usepackage{sidecap}
\usepackage{ulem} 

\parindent = 0 pt
\parskip = 11 pt

\usepackage{amsmath}
\usepackage{amsfonts}
\usepackage{amssymb}
\pagestyle{empty}



%links
\usepackage{hyperref}
\hypersetup{
    colorlinks,%
    citecolor=blue,%
    filecolor=blue,%
    linkcolor=blue,%
    urlcolor=blue
}

\def\dashuline{\bgroup 
\ifdim\ULdepth=\maxdimen  % Set depth based on font, if not set already
\settodepth\ULdepth{(j}\advance\ULdepth.4pt\fi
\markoverwith{\kern.15em
\vtop{\kern\ULdepth \hrule width .3em}%
\kern.15em}\ULon}


\addtolength{\oddsidemargin}{-1in}
\addtolength{\textwidth}{2in}


\begin{document}
\pagestyle{headings}

\newpage

\materia{Teor\'ia de Lenguajes}
\submateria{Segundo Cuatrimestre del 2011}
\titulo{Trabajo Pr\'actico}


\integrante{Ezequiel Castellano}{161/08}{ezequielcastellano@gmail.com}
\integrante{Mariano Semelman}{143/08}{ marianosemelman@gmail.com}

\maketitle

\newpage
\tableofcontents
\newpage
\newpage
\section{Gram\'atica}

Optamos por describir la gram\'atica de manera tal que no tenga conflictos de ning�n tipo (reduce/reduce o shift/reduce). A su vez no perdimos ni agregamos poder de expresividad, ya que de esta manera podemos representar las mismas cadenas que antes. 

La gram\'atica utilizada es la siguiente:

\lstinputlisting{./gramatica.}
\bigskip
\bigskip

\section{Implementaci\'on}

\section{Informaci\'on y requerimientos de software}

En esta secci\'on se indicar\'an versiones, herramientas, compiladores y todo lo necesario para la realizaci\'on del trabajo pr\'actico.

\subsection{Requerimientos}

//TODO: Completar que librerias usamos. Se que son las de C++ y esas cosas, pero no se ni como se dice y no quiero hacer lio, jeje. 

Es necesario tener instaladas las librerias para compilar en c++ (GNU GCC)y Bison. El trabajo pr\'actico fue implementado sobre Ubuntu 11.05.

\subsection{Compilar}

Utilizar el MakeFile provisto con el c\'odigo, el cual se encuentra detallado en el ap\'endice.

\section{Casos de Prueba}

//TODO: Estos fueron los que usamos al comienzo para ver que la gram\'atica sea v\'alida, falta completar en que casos la deber\'ia aceptar y en cuales no. 

Las casos de prueba utilizados para validar la gram\'atica fueron los siguientes:

\lstinputlisting{../src/tests/test-regexp.}
\bigskip
\bigskip

\section{Resultados}

//TODO: Completar cuales fueron los resultados obtenidos. 

\section{Conclusiones}

//TODO: Completar las conclusiones. 

Nos fue muy �til para chequear que la gram\'atica no tenga conflictos el tener una herramienta como Bison. 

\section{Ap\'endice}

\subsection{MakeFile}

\lstinputlisting{../src/MakeFile.}
\bigskip
\bigskip

\end{document}
