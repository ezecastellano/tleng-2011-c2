\section{Información y requerimientos de software}

En esta sección se indicarán versiones, herramientas, compiladores y todo lo 
necesario para la realización del trabajo práctico.

\subsection{Requerimientos}

El trabajo práctico fue implementado sobre Ubuntu 11.04. Las siguientes instrucciones
 son válidas en cualquier sistema operativo Linux. De utilizar otro sistema 
 operativo realizar los pasos análogos.

Es necesario tener instalados los paquetes para compilar en \texttt{C++} y \texttt{Bison}.

Para instalar los paquetes de Build Essential (incluye librerias para compilar C++): \texttt{sudo apt-get install build-essential manpages-dev}.

Para instalar los paquetes de Bison: \texttt{sudo apt-get install bison bison-doc}.

(Opcional): Para poder renderizar los archivos .dot es necesario tener instalado el ejecutable dot. 

Para instalarlo: \texttt{sudo apt-get install graphviz}.

\subsection{Compilar}

Utilizar el MakeFile provisto con el código, el cual se encuentra detallado en el apéndice.

\begin{itemize}

 \item Para compilar solamente el ejecutable principal (grep-line): \texttt{make}.

 \item Para compilar los tests: \texttt{make build-test}.
 
 \item Para compilar todo: \texttt{make all}.

 \item Para compilar los tests y ejecutarlos: \texttt{make test}.

 \item Para remover los archivos compilados: \texttt{make clean}. 

 \item Para generar el gráfico de la gramática: \texttt{make bison} (La imagen se encontrará en ./src/graph/grammar.png).

\end{itemize}

\subsection{Ejecutar}

Para ejecutar el programa se deber estar posicionado en el directorio donde 
se encuentre el ejecutable, es decir el directorio donde se compiló.

El archivo ejecutable se llama grep-line y para utilizarlo deber escribirse 
la siguiente sentencia: ./grep-line regexp [file] [-i].

El primer parámetro es la expresión regular, mientras que el segundo parámetro 
opcional es un archivo con las cadenas a matchear, en caso de no utilizarse 
leerá la entrada por entrada estándar.

Para ver gráficos del autómata generado se deber agregar el flag -i. Esta opción
no funciona si no se pasa el archivo como parámetro. El archivo con la información
para graficar aparecerá en la carpeta \texttt{graph/dots/}, para transformar dicho
archivo a png usar el siguiente comando: \\
\texttt{dot graph/dots/grep-line.dot -Tpng -o archivo\_salida.png}
