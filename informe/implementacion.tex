\section{implementación}

La implementación fue realizada en el lenguaje \texttt{C++}, utilizando la herramienta \texttt{Bison}. 

Para recononocer las cadenas enunciadas por la expresión regular decidimos construir 
un autómata finito determinístico al parsear la expresión regular ingresada.
Al construir el autómata inicialmente utilizamos transiciones lambda y permitimos 
la generación de autómatas no determinísticos (dos estados alcanzables por una misma
transición). Al terminar de parsear tomamos el autómata y le aplicamos el algorítmo
de determinización por clausura lambda. 

Para la implementación utilizamos la clase \texttt{Automata}, que internamente usa un 
grafo dirigido y etiquetado, de ahí su nombre \texttt{LDGraph} (Labeled Directed Graph).

En la clase Autómata se implementaron todos los métodos necesarios para poder 
concatenar autómatas, determinizarlos, mostrarlos, aplicar un operador (*, + o ?) 
o matchear un string. Es importante que para poder matchear el autómata tiene como
precondición estar determinizado. Cualquier operación de concatenación u 
aplicación de operadores transforma el autómata a un no determinístico.

Por otra parte la clase \texttt{LDGraph} consta de un estado de inicial, un conjunto de 
estados finales y un vector de transiciones como estructura. Donde cada transición
es un diccionario, en el cual la clave es la etiqueta de la transición y el 
significado es el conjunto de estados que se pueden alcanzarse por esa transición.

El grafo permite ver si se puede mover de un estado a otro a través de una transición
etiquetada con un determinado caracter, moverse por una transición, determinizarse
y ver si se encuentra en un estado de aceptación, entre sus operaciones más destacables.

Para ver más detalle sobre las estructuras en el apéndice se encuentra detallada 
la implementación.
